\iffalse 
https://tex.stackexchange.com/questions/287924/using-dates-as-section-numbering 
https://www.sharelatex.com/learn/Lists
\fi

\documentclass{article}
\usepackage[english]{babel}
\usepackage[calc,useregional,showdow]{datetime2}
\usepackage{titletoc}
\dottedcontents{section}[1.5in]{\bfseries\large}{1.5in}{10pt}

\usepackage{xparse}
\NewDocumentCommand{\datesec}{o m m}{%
	\renewcommand\thesection{\DTMdate{#2}}
	\IfNoValueTF{#1}
	{\section{#3}}
	{\section[#1]{#3}}
}

\title{Diary Entries}
\author{Eric Morgan - 26863426}

\begin{document}
	\maketitle
	
	\datesec{2018-01-24}{Setup}
	\begin{itemize}
		\item Installed and and started reviewing LaTeX
	\end{itemize}
	
	\datesec{2018-01-25}{Planning}
	\begin{itemize}
		\item Discussed how to go about documentation with Viktoriya and made a 						tentative plan for dividing the work.
	\end{itemize}
	
	\datesec{2018-01-27}{Eclipse}
	\begin{itemize}
		\item Refreshed on Git and Eclipse, installed both.
		\item Troubleshot errors with running project locally in Eclipse.
		\item Reviewed course notes on the parts of the document I'm responsible for and 			started planning them out.
	\end{itemize}
	
	\datesec{2018-01-31}{Use Cases}
	\begin{itemize}
		\item Discussed with team members all the use cases we were going to have in 					iteration 1.
		\item Also discussed what our technical requirements were.
	\end{itemize}
	
	\datesec{2018-02-04}{Documentation}
	\begin{itemize}
		\item Consulted ShareLatex documentation to see how to make tables to display 					the use cases in.
		\item Added non-technical requirements and some of the Use Cases to the 						documentation.
	\end{itemize}
	
	\datesec{2018-02-06}{Diary Formatting}
	\begin{itemize}
		\item Updated diary into the LaTeX template made by Kai.
	\end{itemize}
	
	\datesec{2018-02-07}{More Documentation}
	\begin{itemize}
		\item Wrote up the functional requirements for the documentation.
		\item Wrote the main flow for each of the use cases.
		\item Clarified the requirements for use cases, use diagram, and business rules 				with the TA, and subsequently planned new use cases to include while 						condensing some of the ones I had already written.
	\end{itemize}
	
	\datesec{2018-02-08}{Business Rules and Diagrams}
	\begin{itemize}
		\item Changed some of the use cases to apply to features we plan to develop in 					future iterations.
		\item Wrote business rules that apply to the use cases.
		\item Created a use case diagram that encompasses all the proposed use cases.
	\end{itemize}
	
	\datesec{2018-02-28}{Planning Iteration 2}
	\begin{itemize}
		\item Discussed use cases for iteration two. I'll be working on adding a 						categories component of each transaction entry that will allow the user to 					get a better understanding of their spending habits by being able to view 					all their transactions broken down by category. 
	\end{itemize}
	
	\datesec{2018-03-02}{Reviewing Code and Git}
	\begin{itemize}
		\item Began reviewing the code as it currently exists after iteration one.
		\item Reviewed basic Git commands such as branching and merging to prepare to 					start adding to the code. 
	\end{itemize}
	
	\datesec{2018-03-03}{Adding New Fields}
	\begin{itemize}
		\item Added a new field "Type" to the database for transactions to be used later 			for sorting.
		\item Researched and discussed with Matt best ways of categorizing a database 					table. 
	\end{itemize}
	
	\datesec{2018-03-07}{New Model/DAO for Type}
	\begin{itemize}
		\item Began implementation of the Type database table, basing the classes of 					those already created for Transactions and Accounts.
		\item Utilizing the ForeignCollections property of OrmLite, I got a successful 					test working of creating a new database table of Types that contains a 							collections of Transactions that are all of that Type.
		\item After a lot of troubleshooting, I was able to get all transactions to display 			grouped by type in the transactions viewing window.
	\end{itemize}
	
	\datesec{2018-03-12}{Cleaning Up}
	\begin{itemize}
		\item Fixed an issue where the sorted transactions would not have their JavaFX onClick 		methods working properly.
		\item Edited the CSS to displays the lists more clearly. 
	\end{itemize}
	
\end{document}