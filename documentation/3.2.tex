\documentclass[12pt]{article}

\pagestyle{empty}
\setcounter{secnumdepth}{2}

\usepackage{graphicx}
\usepackage{xcolor,colortbl, wrapfig}
\usepackage{array}
\iffalse 
https://tex.stackexchange.com/questions/94799/how-do-i-color-table-columns
\fi

\newcolumntype{a}{>{\columncolor{Gray}}c}
\newcolumntype{b}{>{\columncolor{white}}c}

\topmargin=0cm
\oddsidemargin=0cm
\textheight=22.0cm
\textwidth=16cm
\parindent=0cm
\parskip=0.15cm
\topskip=0truecm
\raggedbottom
\abovedisplayskip=3mm
\belowdisplayskip=3mm
\abovedisplayshortskip=0mm
\belowdisplayshortskip=2mm
\normalbaselineskip=12pt
\normalbaselines

\begin{document}

\vspace*{0.5in}
\centerline{\bf\Large Design Document}

\vspace*{0.5in}
\centerline{\bf\Large Team X}

\vspace*{0.5in}
\centerline{\bf\Large 7 february 2012}

\vspace*{1.5in}
\begin{table}[htbp]
\caption{Team}
\begin{center}
\begin{tabular}{|r | c|}
\hline
Name & ID Number \\
\hline\hline
X & Y \\
\hline
\end{tabular}
\end{center}
\end{table}

\clearpage

\section{Introduction}

The introduction of the document provides an overview of the entire document,
briefly introducing what are its goals, and what information is to be found in it.

\section{Architectural Design} \label{sec:arch}

This section must give a high-level description of the system in terms of its modules
and their respective purpose and exact interfaces.

\subsection{Architectural Diagram}

A UML class diagram or package diagram depicting the high-level structure of the system,
accompanied by a one-paragraph text describing the rationale of this design.
It is mandatory that the system be divided into at least two subsystems,
and that the purpose of each of these subsystems be exposed here.

\subsection{Subsystem Interface Specifications}

Specification of the software interfaces between the subsystems,
i.e. specific messages (or function calls) that are exchanged by the subsystems.
These are also often called ``Module Interface Specifications''.
Description of the parameters to be passed into these function calls in order to have a service fulfilled,
including valid and invalid ranges of values.
Each subsystem interface must be presented in a separate subsection.

\section{Detailed Design} \label{sec:detail}

Complete description of the system design, describing one subsystem separately in respective subsection.
UML class diagrams are to be used, as well as a short textual description describing the purpose of each class.

\subsection{Drop Down Menu - Categories}

\subsubsection{Detailed Design Diagram}

UML class diagram depicting the internal structure of the subsystem,
accompanied by a paragraph of text describing the rationale of this design.

\subsubsection{Units Description}

\subsubsection{Detailed Design Diagram}

\subsubsection{Units Description}

\newcommand{\mc}[2]{\multicolumn{#1}{c}{#2}}
\newcolumntype{L}{>{\centering\arraybackslash}m{3cm}}
\definecolor{Gray}{gray}{0.85}
\definecolor{LightCyan}{rgb}{0.88,1,1}

\setcounter{secnumdepth}{4}

\subsubsection{Statistics$\langle Scene \rangle$}
\begin{table}
	\begin{tabular}{|a|p{15cm}|}
		\hline
		\rowcolor{LightCyan}
		\mc{0}{Class Name} & \mc{0}{Statistics} \\
		\hline
		Description & User experience depends on tools to properly visualize monetary data. This class contains methods to process data for being used in display classes. Using this class involves calls of getters to acquire processed data. \\
		\hline
			Attributes & 
					 	\begin{tabular}{| p{2cm} | p{2cm} | p{3cm} | p{6.45cm} |}
							\hline
							\rowcolor{gray}
							Visibility & Data Type & Name & Description \\
						\end{tabular} \\
					\hline
		 Methods & 		 
							 	\begin{tabular}{| p{2cm} | p{5cm} | p{6.9cm} |}
							 		\hline
							 		\rowcolor{gray}
							 		\mc{1}{Visibility} &\mc{1}{Name} & \mc{1}{Description} \\
						 			\hline
						 			\rowcolor{white}
						 			public &  getAllTransactions() & Return the transaction Dao records\\
						 			\hline
						 			public &  getTotalMonths() & Return the total months of records\\
						 									 			\hline
						 			public &  getMonthsTransactions(int number) & Return the Transaction list matching a time\\
						 									 			\hline
						 			public &  getTotalAverage() & Return the average of all transactions\\
						 									 			\hline
						 			public &  getMonthsAverage(int number) &  Return the average of all transactions in a month\\
						 									 			\hline
						 			public &  getAverageOut() & Return the average of all negative transactions\\
						 									 			\hline
						 			public &  getAverageInMonth(int number) & Return the average in a month\\
						 									 			\hline
						 			public &  getAverageOutMonth(int number) & Return the average of all negative transactions in a month\\
						 									 			\hline
						 			public &  getTotalMedian() & Return the median of all\\
						 									 			\hline
						 			public &  getMonthsMedian(int number) & Return the median of months\\
						 									 			\hline
						 			public &  getMedianIn() & \\
						 									 			\hline
						 			public &  getMedianOut() & \\
						 									 			\hline
						 			public &  getMedianInMonth(int number) & \\
						 									 			\hline
						 			public &  getMedianOutMonth(int number) & \\
						 									 			\hline
						 			public &  getRecurring() & \\
						 									 			\hline
						 			public &  getMaxIn() & Return the largest tranaction\\
						 									 			\hline
						 			public &  getMaxOut() & Return the max withdrawn\\
						 									 			\hline
						 			public &  getMaxRecurring() & \\
						 									 			\hline
						 			
							 	\end{tabular}								 
	\end{tabular}
\end{table}

Methods:
\begin{table}
	\begin{tabular}{|a|p{5cm}|}
		\hline
		\rowcolor{LightCyan}
		Method Name & getAllTransactions()\\
		Description & desc\\
		Input & None \\
		Output & List of transactions\\
Return Type & Generic type of Dao  \\
		
	\end{tabular}
\end{table}

\begin{table}
	\begin{tabular}{|a|p{5cm}|}
		\hline
		\rowcolor{LightCyan}
		Method Name & getTotalMonths()\\
		Description &  Return the total months of records\\
		Input & None \\
		Output & Total months \\
		Return Type & Integer \\
	\end{tabular}
\end{table} 

\begin{table}
	\begin{tabular}{|a|p{5cm}|}
		\hline
		\rowcolor{LightCyan}
		Method Name & getMonthsTransactions(int number)\\
		Description & Return the Transaction list matching a time\\
		Input & integer month number \\
		Output & List of transactions\\
		Return Type & Generic type of Dao  \\
		
	\end{tabular}
\end{table}

\begin{table}
	\begin{tabular}{|a|p{5cm}|}
		\hline
		\rowcolor{LightCyan}
		Method Name & getTotalAverage()\\
		Description &  Return the average of all transactions\\
		Input & None \\
		Output & Output average \\
		Return Type & double \\
		
	\end{tabular}
\end{table}

\begin{table}
	\begin{tabular}{|a|p{5cm}|}
		\hline
		\rowcolor{LightCyan}
		Method Name & getMonthsAverage(int number)\\
		Description & return the average of all transactions in a month\\
		Input & int month number \\
		Output & average at a time \\
		Return Type & double \\
		
	\end{tabular}
\end{table}

\begin{table}
	\begin{tabular}{|a|p{5cm}|}
		\hline
		\rowcolor{LightCyan}
		Method Name & getAverageIn()\\
		Description &  Return the average of all possitive transactions\\
		Input & None \\
		Output & positive average \\
		Return Type & double \\
		
	\end{tabular}
\end{table}

\begin{table}
	\begin{tabular}{|a|p{5cm}|}
		\hline
		\rowcolor{LightCyan}
		Method Name & getAverageOut()\\
		Description & Return the average of all negative transactions\\
		Input & None \\
		Output & negative transaction average \\
		Return Type & double \\
		
	\end{tabular}
\end{table}

\begin{table}
	\begin{tabular}{|a|p{5cm}|}
		\hline
		\rowcolor{LightCyan}
		Method Name & getAverageInMonth(int number)\\
		Description & Return the average in a month\\
		Input & int month number \\
		Output & average in a month \\
		Return Type & double \\
		
	\end{tabular}
\end{table}

\begin{table}
	\begin{tabular}{|a|p{5cm}|}
		\hline
		\rowcolor{LightCyan}
		Method Name & getAverageOutMonth(int number)\\
		Description & Return the average of all negative transactions in a month\\
		Input & integer month number \\
		Output & average aquired in month\\
		Return Type & double \\	
	\end{tabular}
\end{table}

\begin{table}
	\begin{tabular}{|a|p{5cm}|}
		\hline
		\rowcolor{LightCyan}
		Method Name & getTotalMedian()\\
		Description & desc\\
		Input & None \\
		Output & None \\
		Return Type & None \\
		
	\end{tabular}
\end{table}

\begin{table}
	\begin{tabular}{|a|p{5cm}|}
		\hline
		\rowcolor{LightCyan}
		Method Name & getTotalMedian()\\
		Description & Return the median of all\\
		Input & - \\
		Output & - \\
		Return Type & - \\
		
	\end{tabular}
\end{table}

\begin{table}
	\begin{tabular}{|a|p{5cm}|}
		\hline
		\rowcolor{LightCyan}
		Method Name & getMonthsMedian(int number)\\
		Description & Return the median of months\\
		Input & integer month number \\
		Output & median of month\\
		Return Type & double \\	

		
	\end{tabular}
\end{table}

\begin{table}
	\begin{tabular}{|a|p{5cm}|}
		\hline
		\rowcolor{LightCyan}
		Method Name & getMedianIn()\\
		Description & desc\\	
		Input & - \\
		Output & - \\
		Return Type & - \\		
	\end{tabular}
\end{table}

\begin{table}
	\begin{tabular}{|a|p{5cm}|}
		\hline
		\rowcolor{LightCyan}
		Method Name & getMedianOut()\\
		Description & desc\\	
Input & - \\
Output & - \\
Return Type & - \\	
		
	\end{tabular}
\end{table}

\begin{table}
	\begin{tabular}{|a|p{5cm}|}
		\hline
		\rowcolor{LightCyan}
		Method Name & getMedianInMonth()\\
		Description & desc\\	
Input & - \\
Output & - \\
Return Type & - \\	
		
	\end{tabular}
\end{table}

\begin{table}
	\begin{tabular}{|a|p{5cm}|}
		\hline
		\rowcolor{LightCyan}
		Method Name & getMedianOutMonth()\\
		Description & desc\\	
Input & - \\
Output & - \\
Return Type & - \\	
		
	\end{tabular}
\end{table}

\begin{table}
	\begin{tabular}{|a|p{5cm}|}
		\hline
		\rowcolor{LightCyan}
		Method Name & getRecurring()\\
		Description & desc\\
		Input & None \\
		Output & None \\
		Return Type & None \\
		
	\end{tabular}
\end{table}

\begin{table}
	\begin{tabular}{|a|p{5cm}|}
		\hline
		\rowcolor{LightCyan}
		Method Name & getMaxIn()\\
		Description & Return the largest tranaction\\
		Input & None \\
		Output &  single Transaction \\
		Return Type & Transaction  \\
		
	\end{tabular}
\end{table}

\begin{table}
	\begin{tabular}{|a|p{5cm}|}
		\hline
		\rowcolor{LightCyan}
		Method Name & getMaxOut()\\
		Description & Return the max withdrawn\\
		Input & None \\
		Output &  single Transaction \\
		Return Type & Transaction  \\
		
	\end{tabular}
\end{table}

\begin{table}
	\begin{tabular}{|a|p{5cm}|}
		\hline
		\rowcolor{LightCyan}
		Method Name & getMaxRecurring()\\
		Description & desc\\
		Input & - \\
		Output & - \\
		Return Type & - \\
		
	\end{tabular}
\end{table}

\subsubsection{Transaction $\langle Scene\rangle$:}
\begin{table}
	\begin{tabular}{|a|p{15cm}|}
		\hline
		\rowcolor{LightCyan}
		\mc{0}{Class Name} & \mc{0}{Transaction(String name, String type, double amount, String description, Date date)} \\
		\hline
		Description & The Transaction class holds information on financial data that is to be sent to the server, processed by statistics, deleted or anything requiring data manipulation. \\
		\hline
		Attributes & 
		\begin{tabular}{| p{2cm} | p{2cm} | p{3cm} | p{6.45cm} |}
			\hline
			\rowcolor{gray}
			Visibility & Data Type & Name & Description \\
			
			Private & int & id & an id of the transaction \\
			Private & String & name & Name of user\\
			Private & String & type & the type of action done \\
			Private & double & amount & the quantity acted on \\
			Private & String & description & a description of what it is \\
			Private & Date & date & The date of the transaction \\
		\end{tabular} \\
		\hline
		Methods & 		 
		\begin{tabular}{| p{2cm} | p{5cm} | p{6.9cm} |}
			\hline
			\rowcolor{gray}
			\mc{1}{Visibility} &\mc{1}{Name} & \mc{1}{Description} \\
			\hline
			\rowcolor{white}
			public &  Transaction() & default constructor\\
			\hline
			public &  Transaction(String name, String type, double amount, String description, Date date) & argument constructor\\
			\hline
			public & Transaction(int id, String name, String type, double amount, String description, Date date) & argument constructor with given id\\
			\hline
			public &   getId() & return ID\\
			\hline
			public &  setId(int id) &  set the ID\\
			\hline
			public &  getName()  & get the name\\
			\hline
			public &  setName(String name)  & set the name\\
			\hline
			public &  getType() & get the type\\
			\hline
			public &  setType(String type) & set the type\\
			\hline
			public &  getDescription() & get the description\\
			\hline
			public & setDescription(String description) & set the description\\
			\hline
			public &  getDate() & get the date\\
			\hline
			public &  setDate(Date date)  & set the date\\
			\hline
			public &  getAmount() & get the amount\\
			\hline
			public &   setAmount(double amount) & set the amount\\
			\hline
			public &  toString() & Return a string representation of data\\
			\hline
			
		\end{tabular}								 
	\end{tabular}
\end{table}
Methods:
\begin{table}
	\begin{tabular}{|a|p{5cm}|}
		\hline
		\rowcolor{LightCyan}
		Method Name &  Transaction()\\ 
		Description & default constructor\\
		Input & None \\
		Output & None\\
		Return Type & None \\
		
	\end{tabular}
\end{table}

\begin{table}
	\begin{tabular}{|a|p{5cm}|}
		\hline
		\rowcolor{LightCyan}
		Method Name &  Transaction()\\ 
Description & default constructor\\
Input & None \\
Output & None\\
Return Type & None \\
	\end{tabular}
\end{table} 

\begin{table}
	\begin{tabular}{|a|p{5cm}|}
		\hline
		\rowcolor{LightCyan}
		Method Name &  Transaction()\\ 
Description & default constructor\\
Input & None \\
Output & None\\
Return Type & None \\
		
	\end{tabular}
\end{table}

\begin{table}
	\begin{tabular}{|a|p{5cm}|}
		\hline
		\rowcolor{LightCyan}
		Method Name &  Transaction()\\ 
Description & default constructor\\
Input & None \\
Output & None\\
Return Type & None \\
		
	\end{tabular}
\end{table}

\begin{table}
	\begin{tabular}{|a|p{5cm}|}
		\hline
		\rowcolor{LightCyan}
		Method Name &  Transaction()\\ 
Description & default constructor\\
Input & None \\
Output & None\\
Return Type & None \\
		
	\end{tabular}
\end{table}

\begin{table}
	\begin{tabular}{|a|p{5cm}|}
		\hline
		\rowcolor{LightCyan}
		Method Name &  Transaction()\\ 
Description & default constructor\\
Input & None \\
Output & None\\
Return Type & None \\
		
	\end{tabular}
\end{table}

\begin{table}
	\begin{tabular}{|a|p{5cm}|}
		\hline
		\rowcolor{LightCyan}
		Method Name &  Transaction()\\ 
Description & default constructor\\
Input & None \\
Output & None\\
Return Type & None \\
		
	\end{tabular}
\end{table}

\begin{table}
	\begin{tabular}{|a|p{5cm}|}
		\hline
		\rowcolor{LightCyan}
		Method Name &  Transaction()\\ 
Description & default constructor\\
Input & None \\
Output & None\\
Return Type & None \\
		
	\end{tabular}
\end{table}

\begin{table}
	\begin{tabular}{|a|p{5cm}|}
		\hline
		\rowcolor{LightCyan}
		Method Name &  Transaction()\\ 
Description & default constructor\\
Input & None \\
Output & None\\
Return Type & None \\
	\end{tabular}
\end{table}

\begin{table}
	\begin{tabular}{|a|p{5cm}|}
		\hline
		\rowcolor{LightCyan}
		Method Name &  Transaction()\\ 
Description & default constructor\\
Input & None \\
Output & None\\
Return Type & None \\
		
	\end{tabular}
\end{table}

\begin{table}
	\begin{tabular}{|a|p{5cm}|}
		\hline
		\rowcolor{LightCyan}
		Method Name &  Transaction()\\ 
Description & default constructor\\
Input & None \\
Output & None\\
Return Type & None \\
		
	\end{tabular}
\end{table}

\begin{table}
	\begin{tabular}{|a|p{5cm}|}
		\hline
		\rowcolor{LightCyan}
		Method Name &  Transaction()\\ 
Description & default constructor\\
Input & None \\
Output & None\\
Return Type & None \\
		
	\end{tabular}
\end{table}

\begin{table}
	\begin{tabular}{|a|p{5cm}|}
		\hline
		\rowcolor{LightCyan}
		Method Name &  Transaction()\\ 
Description & default constructor\\
Input & None \\
Output & None\\
Return Type & None \\	
	\end{tabular}
\end{table}

\begin{table}
	\begin{tabular}{|a|p{5cm}|}
		\hline
		\rowcolor{LightCyan}
		Method Name &  Transaction()\\ 
Description & default constructor\\
Input & None \\
Output & None\\
Return Type & None \\
		
	\end{tabular}
\end{table}

\begin{table}
	\begin{tabular}{|a|p{5cm}|}
		\hline
		\rowcolor{LightCyan}
		Method Name &  Transaction()\\ 
Description & default constructor\\
Input & None \\
Output & None\\
Return Type & None \\
		
	\end{tabular}
\end{table}

\begin{table}
	\begin{tabular}{|a|p{5cm}|}
		\hline
		\rowcolor{LightCyan}
		Method Name &  Transaction()\\ 
Description & default constructor\\
Input & None \\
Output & None\\
Return Type & None \\
		
	\end{tabular}
\end{table}

\begin{table}
	\begin{tabular}{|a|p{5cm}|}
		\hline
		\rowcolor{LightCyan}
		Method Name &  Transaction()\\ 
Description & default constructor\\
Input & None \\
Output & None\\
Return Type & None \\
		
	\end{tabular}
\end{table}

\begin{table}
	\begin{tabular}{|a|p{5cm}|}
		\hline
		\rowcolor{LightCyan}
		Method Name &  Transaction()\\ 
Description & default constructor\\
Input & None \\
Output & None\\
Return Type & None \\
		
	\end{tabular}
\end{table}

\begin{table}
	\begin{tabular}{|a|p{5cm}|}
		\hline
		\rowcolor{LightCyan}
		Method Name &  Transaction()\\ 
Description & default constructor\\
Input & None \\
Output & None\\
Return Type & None \\
		
	\end{tabular}
\end{table}

\begin{table}
	\begin{tabular}{|a|p{5cm}|}
		\hline
		\rowcolor{LightCyan}
		Method Name &  Transaction()\\ 
Description & default constructor\\
Input & None \\
Output & None\\
Return Type & None \\
		
	\end{tabular}
\end{table}
\pagebreak


\subsubsection{TransactionDao $\langle Scene\rangle$:}
\begin{table}
	\begin{tabular}{|a|p{15cm}|}
		\hline
		\rowcolor{LightCyan}
		\mc{0}{Class Name} & \mc{0}{Transaction(String name, String type, double amount, String description, Date date)} \\
		\hline
		Description & A transaction Dao is the way that transaction connect to the database. a Transaction Dao has reference to a set of Transaction objects and performs operations on networks with them.\\
		\hline
		Attributes & 
		\begin{tabular}{| p{2cm} | p{3.5cm} | p{1.5cm} | p{6.45cm} |}
			\hline
			\rowcolor{gray}
			Visibility & Data Type & Name & Description \\
			private & ConnectionSource & connection & The address of a connection \\
			private & Dao $\langle Transaction, Integer \rangle$ & transactionDao & the Transaction it is connected to \\
			
		\end{tabular} \\
		\hline
		Methods & 		 
		\begin{tabular}{| p{2cm} | p{5cm} | p{6.9cm} |}
			\hline
			\rowcolor{gray}
			\mc{1}{Visibility} &\mc{1}{Name} & \mc{1}{Description} \\
			\hline
			\rowcolor{white}			
			public &  TransactionDao() & Default constructor\\
			\hline
			public &  getTransactionById() & Return the record with a certain ID\\
			\hline
			public &  getAllTransactions(int number) & Return all Transaction objects contained\\
			\hline
			public &  updateTransaction(Transaction transaction) & updates an sql record\\
			\hline
			public &  insert(Transaction transaction) &  inserts a transaction\\
			\hline
			public &  delete(Transaction transaction) & Removes a transaction\\
			\hline
		
		\end{tabular}								 
	\end{tabular}
\end{table}


Methods:
\begin{table}
	\begin{tabular}{|a|p{5cm}|}
		\hline
		\rowcolor{LightCyan}
		Method Name & TransactionDao()\\
		Description & desc\\
		Input & None \\
		Output & List of transactions\\
		Return Type & Generic type of Dao  \\
		
	\end{tabular}
\end{table}

\begin{table}
	\begin{tabular}{|a|p{5cm}|}
		\hline
		\rowcolor{LightCyan}
		Method Name & getTransactionById()\\
		Description &  Return the total months of records\\
		Input & None \\
		Output & Total months \\
		Return Type & Integer \\
	\end{tabular}
\end{table} 

\begin{table}
	\begin{tabular}{|a|p{5cm}|}
		\hline
		\rowcolor{LightCyan}
		Method Name & getAllTransactions(int number)\\
		Description & Return the Transaction list matching a time\\
		Input & integer month number \\
		Output & List of transactions\\
		Return Type & Generic type of Dao  \\
		
	\end{tabular}
\end{table}

\begin{table}
	\begin{tabular}{|a|p{5cm}|}
		\hline
		\rowcolor{LightCyan}
		Method Name & updateTransaction(Transaction transaction)\\
		Description &  Return the average of all transactions\\
		Input & None \\
		Output & Output average \\
		Return Type & double \\
		
	\end{tabular}
\end{table}

\begin{table}
	\begin{tabular}{|a|p{5cm}|}
		\hline
		\rowcolor{LightCyan}
		Method Name & insert(Transaction transaction)\\
		Description & return the average of all transactions in a month\\
		Input & int month number \\
		Output & average at a time \\
		Return Type & double \\
		
	\end{tabular}
\end{table}

\begin{table}
	\begin{tabular}{|a|p{5cm}|}
		\hline
		\rowcolor{LightCyan}
		Method Name & delete(Transaction transaction)\\
		Description &  Return the average of all possitive transactions\\
		Input & None \\
		Output & positive average \\
		Return Type & double \\
		
	\end{tabular}
\end{table}

\pagebreak

\subsubsection{TransactionView $\langle Scene\rangle$:}
\begin{table}
	\begin{tabular}{|a|p{15cm}|}
		\hline
		\rowcolor{LightCyan}
		\mc{0}{Class Name} & \mc{0}{Transaction(String name, String type, double amount, String description, Date date)} \\
		\hline
		Description & User experience depends on tools to properly visualize monetary data. This class contains methods to process data for being used in display classes. Using this class involves calls of getters to acquire processed data. \\
		\hline
		
		Attributes & 
		\begin{tabular}{| p{2cm} | p{2cm} | p{3cm} | p{6.45cm} |}
			\hline
			\rowcolor{gray}
			Visibility & Data Type & Name & Description \\
			private & Transaction & transaction  & Description \\
			private & EventHandler$\rangle MouseEvent\rangle$  & onAction  & Description\\
			private & ObjectProperty$\rangle EventHandler\rangle MouseEvent\rangle\rangle$ & propertyOnAction & Description \\
		\end{tabular} \\
		\hline
		Methods & 		 
		\begin{tabular}{| p{2cm} | p{5cm} | p{6.9cm} |}
			\hline
			\rowcolor{gray}
			\mc{1}{Visibility} &\mc{1}{Name} & \mc{1}{Description} \\
			\hline
			\rowcolor{white}			
			public &  TransactionView(Transaction transaction, EventHandler$\langle MouseEvent\rangle$ onAction)  & default constructor\\
			\hline
			public &  styleComponent() & Return the total months of records\\
			\hline
			public &  setContent(int number) & Return the Transaction list matching a time\\
			\hline
			public &  setOnAction(EventHandler$\langle MouseEvent\rangle$ eventHandler)  & Return the average of all transactions\\
			\hline
			public &  getOnAction(int number) &  Return the average of all transactions in a month\\
			\hline
			public &  getTransaction() & Return the average of all negative transactions\\
			\hline
		\end{tabular}								 
	\end{tabular}
\end{table}
Methods:
\begin{table}
	\begin{tabular}{|a|p{5cm}|}
		\hline
		\rowcolor{LightCyan}
		Method Name & getAllTransactions()\\
		Description & desc\\
		Input & None \\
		Output & List of transactions\\
		Return Type & Generic type of Dao  \\
		
	\end{tabular}
\end{table}

\begin{table}
	\begin{tabular}{|a|p{5cm}|}
		\hline
		\rowcolor{LightCyan}
		Method Name & getTotalMonths()\\
		Description &  Return the total months of records\\
		Input & None \\
		Output & Total months \\
		Return Type & Integer \\
	\end{tabular}
\end{table} 

\begin{table}
	\begin{tabular}{|a|p{5cm}|}
		\hline
		\rowcolor{LightCyan}
		Method Name & getMonthsTransactions(int number)\\
		Description & Return the Transaction list matching a time\\
		Input & integer month number \\
		Output & List of transactions\\
		Return Type & Generic type of Dao  \\
		
	\end{tabular}
\end{table}

\begin{table}
	\begin{tabular}{|a|p{5cm}|}
		\hline
		\rowcolor{LightCyan}
		Method Name & getTotalAverage()\\
		Description &  Return the average of all transactions\\
		Input & None \\
		Output & Output average \\
		Return Type & double \\
		
	\end{tabular}
\end{table}

\begin{table}
	\begin{tabular}{|a|p{5cm}|}
		\hline
		\rowcolor{LightCyan}
		Method Name & getMonthsAverage(int number)\\
		Description & return the average of all transactions in a month\\
		Input & int month number \\
		Output & average at a time \\
		Return Type & double \\
		
	\end{tabular}
\end{table}

\begin{table}
	\begin{tabular}{|a|p{5cm}|}
		\hline
		\rowcolor{LightCyan}
		Method Name & getAverageIn()\\
		Description &  Return the average of all possitive transactions\\
		Input & None \\
		Output & positive average \\
		Return Type & double \\
		
	\end{tabular}
\end{table}

\begin{table}
	\begin{tabular}{|a|p{5cm}|}
		\hline
		\rowcolor{LightCyan}
		Method Name & getAverageOut()\\
		Description & Return the average of all negative transactions\\
		Input & None \\
		Output & negative transaction average \\
		Return Type & double \\
		
	\end{tabular}
\end{table}

\begin{table}
	\begin{tabular}{|a|p{5cm}|}
		\hline
		\rowcolor{LightCyan}
		Method Name & getAverageInMonth(int number)\\
		Description & Return the average in a month\\
		Input & int month number \\
		Output & average in a month \\
		Return Type & double \\
		
	\end{tabular}
\end{table}

\begin{table}
	\begin{tabular}{|a|p{5cm}|}
		\hline
		\rowcolor{LightCyan}
		Method Name & getAverageOutMonth(int number)\\
		Description & Return the average of all negative transactions in a month\\
		Input & integer month number \\
		Output & average aquired in month\\
		Return Type & double \\	
	\end{tabular}
\end{table}

\begin{table}
	\begin{tabular}{|a|p{5cm}|}
		\hline
		\rowcolor{LightCyan}
		Method Name & getTotalMedian()\\
		Description & desc\\
		Input & None \\
		Output & None \\
		Return Type & None \\
		
	\end{tabular}
\end{table}

\begin{table}
	\begin{tabular}{|a|p{5cm}|}
		\hline
		\rowcolor{LightCyan}
		Method Name & getTotalMedian()\\
		Description & Return the median of all\\
		Input & - \\
		Output & - \\
		Return Type & - \\
		
	\end{tabular}
\end{table}

\begin{table}
	\begin{tabular}{|a|p{5cm}|}
		\hline
		\rowcolor{LightCyan}
		Method Name & getMonthsMedian(int number)\\
		Description & Return the median of months\\
		Input & integer month number \\
		Output & median of month\\
		Return Type & double \\	
		
		
	\end{tabular}
\end{table}

\begin{table}
	\begin{tabular}{|a|p{5cm}|}
		\hline
		\rowcolor{LightCyan}
		Method Name & getMedianIn()\\
		Description & desc\\	
		Input & - \\
		Output & - \\
		Return Type & - \\		
	\end{tabular}
\end{table}

\begin{table}
	\begin{tabular}{|a|p{5cm}|}
		\hline
		\rowcolor{LightCyan}
		Method Name & getMedianOut()\\
		Description & desc\\	
		Input & - \\
		Output & - \\
		Return Type & - \\	
		
	\end{tabular}
\end{table}

\begin{table}
	\begin{tabular}{|a|p{5cm}|}
		\hline
		\rowcolor{LightCyan}
		Method Name & getMedianInMonth()\\
		Description & desc\\	
		Input & - \\
		Output & - \\
		Return Type & - \\	
		
	\end{tabular}
\end{table}

\begin{table}
	\begin{tabular}{|a|p{5cm}|}
		\hline
		\rowcolor{LightCyan}
		Method Name & getMedianOutMonth()\\
		Description & desc\\	
		Input & - \\
		Output & - \\
		Return Type & - \\	
		
	\end{tabular}
\end{table}

\begin{table}
	\begin{tabular}{|a|p{5cm}|}
		\hline
		\rowcolor{LightCyan}
		Method Name & getRecurring()\\
		Description & desc\\
		Input & None \\
		Output & None \\
		Return Type & None \\
		
	\end{tabular}
\end{table}

\begin{table}
	\begin{tabular}{|a|p{5cm}|}
		\hline
		\rowcolor{LightCyan}
		Method Name & getMaxIn()\\
		Description & Return the largest tranaction\\
		Input & None \\
		Output &  single Transaction \\
		Return Type & Transaction  \\
		
	\end{tabular}
\end{table}

\begin{table}
	\begin{tabular}{|a|p{5cm}|}
		\hline
		\rowcolor{LightCyan}
		Method Name & getMaxOut()\\
		Description & Return the max withdrawn\\
		Input & None \\
		Output &  single Transaction \\
		Return Type & Transaction  \\
		
	\end{tabular}
\end{table}

\begin{table}
	\begin{tabular}{|a|p{5cm}|}
		\hline
		\rowcolor{LightCyan}
		Method Name & getMaxRecurring()\\
		Description & desc\\
		Input & - \\
		Output & - \\
		Return Type & - \\
		
	\end{tabular}
\end{table}
\pagebreak





\end{document}
